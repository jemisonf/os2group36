\nonstopmode % halt on errors
\documentclass[onecolumn, draftclsnofoot,10pt, compsoc]{IEEEtran}
\usepackage{graphicx}
\usepackage{url}
\usepackage{setspace}
\usepackage{minted}
\usepackage{geometry}
\geometry{textheight=9.5in, textwidth=7in}

% 1. Fill in these details
\def \CapstoneTeamName{		Group 36}
\def \CapstoneTeamNumber{		36}
\def \GroupMemberOne{			Scott Russell}
\def \GroupMemberTwo{			Arya Asgari}
\def \GroupMemberThree{			Fischer Jemison}
\def \CapstoneProjectName{Project 2: I/O Elevators  }



\def \DocType{	
				}
			
\newcommand{\NameSigPair}[1]{\par
\makebox[2.75in][r]{#1} \hfil 	\makebox[3.25in]{\makebox[2.25in]{\hrulefill} \hfill		\makebox[.75in]{\hrulefill}}
\par\vspace{-12pt} \textit{\tiny\noindent
\makebox[2.75in]{} \hfil		\makebox[3.25in]{\makebox[2.25in][r]{Signature} \hfill	\makebox[.75in][r]{Date}}}}
% 3. If the document is not to be signed, uncomment the RENEWcommand below
%\renewcommand{\NameSigPair}[1]{#1}

%%%%%%%%%%%%%%%%%%%%%%%%%%%%%%%%%%%%%%%
\begin{document}
\begin{titlepage}
    \pagenumbering{gobble}
    \begin{singlespace}
        %\includegraphics[height=4cm]{coe_v_spot1}
        \hfill 
        % 4. If you have a logo, use this includegraphics command to put it on the coversheet.
        %\includegraphics[height=4cm]{CompanyLogo}   
        \par\vspace{.2in}
        \centering
        \scshape{
            \huge OS 444 Group 36 \DocType \par
            {\large\today}\par
            \vspace{.5in}
            \textbf{\Huge\CapstoneProjectName}\par
          
           \vfill
            
          
            \vspace{5pt}
       
            {\large Group 36 }\par
    
         \vspace{5pt}
            {\Large
                \par{\GroupMemberOne}\par
               {\GroupMemberTwo}\par
               {\GroupMemberThree}\par
            }
            \vspace{20pt}
    
            

            % 5. comment out the line below this one if you do not wish to name your team
            \vspace{5pt}
        }     
    \end{singlespace}
\end{titlepage}
\newpage
\pagenumbering{arabic}
\tableofcontents
\clearpage


\section{Design Plan}
Our designed stemmed from an open source baseline implementation of the no-op scheduler found here: "https://elixir.bootlin.com/linux/latest/source/block/noop-iosched.c" If you compare that base implementation to ours you can see that very little was changed in the grand scheme of things. Mainly the creation of the Forward/Back head and being able to step through the queue up and down the elevator. We decided to go with the LOOK implementation instead of C-Look for simplicity. Rather than having to only drop off in a single direction and 'drop' down to reset after each elevator lift having the ability to go in both directions made creation the forward/backwards directions in the queue simpler as it was reversible.

\section{Version Control: Table}
  \begin{tabular} {| p{2.0cm} | p{7.4cm}  | p{3.0cm} | }
\hline
\textbf{Commit Hash} & \textbf{Commit Description} & \textbf{Date Added}  \\
\hline bef7896 & Initial sstf-iosched.c file & Fri May 4th 2018 -0700 \\
\hline 182072c & readme for building sstf & Fri May 4th 2018 -0700 \\
\hline adaa8b2 & update readme & Fri May 4th 2018 -0700 \\
\hline 8f4f083 & delete readme.md (old one) & Fri May 4th 2018 -0700 \\
\hline 8273541 & Added documentation for io schedulers & Sat May 5th 2018 -0700 \\

\hline 80c6ffa & Adds test stuff and tweaks README & Sat May 5th 2018 -0700 \\

\hline 70c5fce & removes test files & Sat May 5th 2018 -0700 \\

\hline 9bd63b3 & update sstf-iosched.c &  Sun May 6th 2018 -0700 \\

\hline 0b86c99 & Updates README &  Sun May 6th 2018 -0700 \\

\hline  75c1040 & Adds block.patch file &  Sun May 6th 2018 -0700 \\


\hline
\end {tabular}
\\ \\



\section{Work Log: What was done when?}
\begin{tabular} {| p{2.5cm} | p{8.4cm}|}

\hline
\textbf{May 2nd} & \textbf{Work Done} \\
\hline 
Scott Russell & Researched sstf scheduler and comparison to no-op.\\

\hline 
\hline 
Arya Asgari &  Researched no-op scheduler and built-in elevator \\

\hline 
\hline 
Fischer Jemison & Additional research into sstf scheduler as well as kernel building debugging (problems with file access permissions) \\

\hline 
\end {tabular}
\\ \\

\vspace{.1in}
\noindent
\begin{tabular} {| p{2.5cm} | p{8.4cm}|}

\hline
\textbf{May 5th} & \textbf{Work Done} \\
\hline 
Scott Russell & Worked on implementation of sstf scheduler with ReadMe.  \\

\hline 
\hline 
Arya Asgari & Created patch file of sstf file against no-op scheduler, worked on latex documentation.\\

\hline 
\hline 
Fischer Jemison & Created test file to show functionality of code with many ptread calls. \\

\hline 
\end {tabular}
\\ \\

\section{Assignment Questions}
\subsection{Main Point of Assignment}
It seems like the primary objective of the assignment is to start developing the skills necessary to work with the kernel on a low level. Also, this assignment helps build a better understanding of elevator algorithms and the Linux I/O scheduler. Much of this assignment required researching and understanding the no-op I/O scheduler and how the built in elevator already worked. 
\subsection{Approach to Problem}
Allot of the project was thinking about how to implement the scheduler. Looking between the No-Op elevator that I adjusted there was only about 40-50 lines of code changed. This is an example of a time not task assignment. Conceptually thinking about the problem turned into the longest section. We spent our first entire meeting simply thinking about how we were going to implement the sstf file. A small portion was adjusting files and creating the patch file.
\subsection{Ensuring Correctness}
To test the functionality of our scheduler we make a large number of threads that write one thousand character strings to bog down the I/O functionality and to be able to test our sstf scheduler.
\subsection{What we learned}
We learned a great deal while doing this assignment. Since much of this project required research, we spent a fair bit of time understanding the no-op I/O scheduler on a deeper level. We also wanted to make sure we understood the LOOK elevator algorithm, as well as the built in elevator framework. For checking our implementation, we learned how to create a patch file and apply that to the no-op scheduler. 
\subsection{How TA should evaluate work} To evaluate our work, please see the README in our submission. 
\end{document}